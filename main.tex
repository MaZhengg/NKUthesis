% -*- coding: utf-8 -*-
%%
%%
%%
%%
%%
%%
%%  本模板使用以下方式编译:
%%
%%
%%     1. XeLaTeX
%%
%%  注意:
%%    在编译出错后应先删除 *.toc 和 *.aux 文件,
%%    再进行编译
%%
%%
\documentclass[12pt,openright]{book}
\usepackage{ifxetex}
\ifxetex
  \usepackage[bookmarksnumbered]{hyperref}
\else
  \usepackage[unicode,bookmarksnumbered]{hyperref}
\fi

\usepackage[emptydoublepage]{NKThesis}   % 中文
%\usepackage[emptydoublepage,English]{NKThesis} % 英文

\graphicspath{{figure/}}
%   根据需要选择 biblatex 宏包选项.
\usepackage[backend = biber, defernumbers = true,  sorting=none,  style = nkthesis]{biblatex}
\hypersetup{colorlinks=true,
            pdfborder=0 0 1,
            citecolor=black,
            linkcolor=black}
\usepackage{tikz}
\newcommand{\scite}[1]{\textsuperscript{\cite{#1}}}
%重定义\scite使得参考文献应用成为上标
\renewcommand{\thefootnote}{\fnsymbol{footnote}}
%将脚注用符号表示,避免数字引起歧义


%指定参考文献文件的名称,全名.bib
\addbibresource{nkthesis.bib}
%新建条目分类(category)用于区分被引用和未引用的文献条目
\DeclareBibliographyCategory{cited}
%每执行一次除\nocite之外的\cite类命令,将被被引用的文献加到‘cited’分类中
\AtEveryCitekey{\addtocategory{cited}{\thefield{entrykey}}}

\includeonly{
abstract,
introduction,
chapter2,
chapter3,
manual,
%tikz,
acknowledgements,
references,
appendices,
resume
}

\usepackage{ntheorem}
\theoremheaderfont{\jiacu\songti}
\theorembodyfont{\songti}
\newtheorem{Theorem}{\hskip 2em  定理}[chapter]
\newtheorem{Lemma}[Theorem]{\hskip 2em 引理}
\newtheorem{Corollary}[Theorem]{\hskip 2em 推论}
\newtheorem{Proposition}[Theorem]{\hskip 2em 命题}
\newtheorem{Definition}[Theorem]{\hskip 2em 定义}
\newtheorem{Remark}[Theorem]{\hskip 2em 注}
\newtheorem{Example}[Theorem]{\hskip 2em 例}

\theoremstyle{nonumberplain}
\newtheorem{Proof}[Theorem]{\hskip 2em 证明:}

\begin{document}

%  设置基本信息
%  注意:  逗号`,'是项目分隔符. 如果某一项的值出现逗号, 应放在花括号内, 如 {,}
%
\NKTsetup{%
  论文题目(中文) = 毕业论文模板,
  副标题         = ,
  论文题目(英文) =The Thesis,
  论文作者       = ,
  学号           =,
  指导教师       = ,
  申请学位       = ,
  培养单位       = ,
  学科专业       = ,
  研究方向       = ,
  答辩委员会主席 = 王教授,
  评阅人1         = 王教授\hskip 0.5em王教授\hskip 0.5em王教授 , %可以写三位评阅人姓名
  评阅人2         =王教授\hskip 0.5em王教授,  % 其他的评阅人姓名
  中图分类号     = ,
  UDC            = ,
  学校代码       = 10055,
  密级           = 公开,
                   % 公开 | 限制 | 秘密 | 机密, 若为公开, 不填以下三项
  保密期限       = ,
  审批表编号     = ,
  批准日期       = ,
  论文完成时间   = 二〇一三年三月,
  答辩日期       = ,
  论文类别       = 博士,
                   % 博士 | 学历硕士 | 硕士专业学位 | 高校教师 | 同等学力硕士
  院/系/所       = ,
  专业           = ,
  联系电话       = ,
  Email          = ,
  通讯地址(邮编) = ,
  备注           = }


% -*- coding: utf-8 -*-


\begin{zhaiyao}
测试摘要提出了一种基于分段设计的动力翼伞轨迹规划方法: 借鉴传统翼伞分段归航轨迹设计方法,根据动力翼伞的实际工况,加入了动力翼伞的任务执行阶段。根据各阶段的几何关系,将轨迹规划问题转换为几何参数的寻优问题。采用量子遗传算法对目标函数进行寻优计算, 得出各段轨迹几何参数。考虑到由于飞行高度过低动力翼伞无法到达目标点的情况,将额外的高度作为新的变量加入到目标函数中进行优化,所得结果转换为额外的飞行时间通过动力翼伞的等高飞行进行抵消。仿真分析验证了该设计方法的可行性和有效性。
\end{zhaiyao}

\vskip 4mm
\noindent
{\rmfamily\jiacu 关键词:}
\begin{minipage}[t]{0.88\linewidth}
多智能体系统, 一致性, 编队控制, 间歇控制, 脉冲观测器, 脉冲控制, 线性扩张状态观测器, 非线性动态
\end{minipage}




\begin{abstract}
As a kind of special vehicle, the flexible vehicle uses the flexible fabric as itsmaterial, and has the aerodynamic characteristics that the conventional vehicles donot possess. With the development of aerospace techonology, the research on theflexible vehicle is gradually becoming the hotpot home and abroad. This papermainly studies two kinds of the flexible vehicles: high altitude airship and poweredparafoil. Aiming at the issues of the thermodynamic model, super-pressure controlproject, horizontal trajectory control of the airship, and the dynamic model, trajectoryplanning, trajectory tracking control of the powered parafoil, some viewpoints andmethods are presented, and some research production are obtained.
\end{abstract}

\vskip 4mm
\noindent
{\rmfamily\jiacu Key Words:}
\begin{minipage}[t]{0.88\linewidth}
multi-agent systems (MAS), consensus, formation control, impulsive observer, impulsive control, linear extended state observer (LESO),active disturbance rejection control (ADRC), time-delay, uncertainty,  leader, nonlinear dynamics
\end{minipage}

\tableofcontents
\chapter{绪论}
\section{多智能体系统概述}
\subsection{多智能体系统研究背景}
使得整个群体表现出单个个体所不能达到的行为的内在规律,并且将这些内部工作机制及理论应用到实际中\scite{couzin2002collective};物理学家们则希望运用更为精确的数学模型,通过计算机模拟这些现象并且能够深入直观地解释这些令人讶异的群体行为\scite{tanner2003stable,程代展2004从群集到社会行为控制};另外在控制科学领域,\cite{洪奕光2011多智能体系统动态协调与分布式控制设计,olfati2007consensus,savkin2004coordinated}提出了很有意义的研究方向。
\subsection{智能体系统简介}
智能体的研究起源于20世纪70年代初,文本文本文本。
\subsection{多智能体系统一致性问题研究现状}
除了上面提到的Boid 和Vicsek 两个经典模型,下面列举几种常见的多智能体动力学模型及其控制协议的设计。

对于以下的一阶连续积分器动力学模型:(无编号公式)
\[\dot{x}_i=u_i, i=1,2,\ldots,n,
\]
其中,$x_i\in R^n$是第$i$个智能体的位置状态,$u_i\in R^n$表示施加的控制输入。对于这样的单积分器系统,学者们设计了不同的控制协议(有编号公式),
\begin{equation}
u_i=-\sum_{j\in N_i(t)}a_{ij}(t)(x_i(t)-x_j(t)),
\end{equation}

图\ref{formation}展示了几种多智能体系统编队控制在一些领域的应用。
\begin{figure*}[H]
\begin{centering}
\subfloat[多机器人世界杯足球锦标赛]
{
\begin{centering}
\includegraphics[width=0.45\textwidth]{1.png}
\end{centering}
}
\subfloat[无人机编队]
{
\begin{centering}
\includegraphics[width=0.45\textwidth]{1.png}
\end{centering}
}
\end{centering}
\\
\begin{centering}
\subfloat[多个机械手臂协同作业]
{
\begin{centering}
\includegraphics[width=0.45\textwidth]{1.png}
\end{centering}
}
\subfloat[城市交通控制管理系统]
{
\begin{centering}
\includegraphics[width=0.45\textwidth]{1.png}
\end{centering}
}
\end{centering}
\protect\caption{多智能体系统编队控制在实际中的应用}
\label{formation}
\end{figure*}

\chapter{预备知识}
\section{代数图论}

\begin{equation}
T_h=288.15-0.0065H
\end{equation}

大气压强同温度比的指数变化关系为:
\begin{equation}
P_h=P_0\big(\frac{T_h}{T_0}\big)^{5.256}
\end{equation}
定理测试:
\begin{Theorem}

\begin{equation}
P_h=P_0\big(\frac{T_h}{T_0}\big)^{5.256}
\end{equation}
\end{Theorem}
\begin{Theorem}
测试
\end{Theorem}
\begin{Theorem}
测试
\end{Theorem}
\begin{Lemma}
测试引理
\end{Lemma}
\begin{Lemma}
测试引理
\end{Lemma}
参考文献格式\scite{dai2012performance}
参考文献测试\scite{海松2008航空航天概论,colozza2003initial,zhou2007study}

\chapter{基于观测器的非线性多智能体系统间歇一致性控制}
\section{引言}
一致性问题,作为多智能体系统协调控制的基本问题,近些年来受到众多领域专家学者的关注。一致性问题是指设计合适的分布式控制协议使得多个智能体就某个共同兴趣点达成一致的问题。一致性是多智能体系统协调完成任务的前提,与群集、编队等许多问题密切相关,是最基础的也是近几十年来研究者们持续关注的问题,并且取得了大量优秀的成果。\scite{ran2007comprehensive}。
测试引用参考文献\cite{ran2007comprehensive}\scite{ran2007comprehensive,bloemen2002optimizing,breder1954equations,ding2010distributed,dunbar2006distributed,fiedler1973algebraic,gazi2003stability,ji2005connectedness,6166430,milgram1967small,olfati2006flocking,ren2007multi,vicsek1995novel,watts1998collective,zhang2008ultrafast}.
\begin{figure}[!htbp]%[!htbp]是为了固定图片的位置
\centering
\includegraphics[width=0.8\textwidth]{5.png}
\caption{切换拓扑下多智能体系统的轨迹}
\end{figure}
\begin{Theorem}
测试
\end{Theorem}
\begin{Theorem}
测试
\end{Theorem}
\begin{Theorem}
测试
\end{Theorem}
\begin{Lemma}
测试引理
\end{Lemma}
\begin{Lemma}
测试引理
\end{Lemma}
\begin{Remark}
测试标注,“注”是楷体加粗,然后注的内容是宋体。
\end{Remark}
\setcounter{Theorem}{0}
\begin{Example}
测试例子,“例”是楷体加粗,然后例子的内容是宋体。
\end{Example}
\begin{Example}
测试例子
\end{Example}

\include{manual}
%\include{tikz}
% -*- coding: utf-8 -*-

\def\bibrangedash{-}
\printbibliography [ category = cited]


\include{acknowledgements}
\include{appendices}
% -*- coding: utf-8 -*-


\chapter*{个人简历、在学期间发表的学术论文与研究成果}
\section*{个人简历}
王小二,出生于
\section*{攻读博士期间发表的论文}
\begin{enumerate}
\renewcommand{\labelenumi}{[\theenumi]}
\item Bell F K. A note on the irregularity of graphs[J]. Linear algebra and its applications, 1992, 161: 45-54.
\item Bell F K. A note on the irregularity of graphs[J]. Linear algebra and its applications, 1992, 161: 45-54.
\end{enumerate}
\section*{攻读博士期间参与的科研项目}
\begin{enumerate}
\renewcommand{\labelenumi}{[\theenumi]}
\item 国家自然科学基金资助项目。
\item 国家自然科学基金资助项目。
\end{enumerate}

\end{document}
